\documentclass[12pt,a4paper]{report}
\usepackage[portuguese]{babel}
\usepackage[margin=1in]{geometry}
\usepackage[utf8]{inputenc}
\usepackage{amsmath}
\usepackage{amsthm}
\usepackage{amssymb}
\usepackage{amsfonts}
\usepackage[colorlinks,allcolors=blue]{hyperref}
\usepackage{microtype}
\usepackage{graphicx}

\let\emptyset=\varnothing

\newcommand{\tb}{\textbf}
\newcommand{\tbu}[1]{\tb{\textup{#1}}}
\newcommand{\mb}{\mathbf}

\newcommand{\dpar}[1]{\left(#1\right)}
\newcommand{\dsqr}[1]{\left[#1\right]}
\newcommand{\dcur}[1]{\left\{#1\right\}}
\newcommand{\dabs}[1]{\left|#1\right|}
\newcommand{\ang}[1]{\left\langle#1\right\rangle}

\newcommand{\ds}{\displaystyle}

\newcommand{\N}{\mathbb{N}}
\newcommand{\Z}{\mathbb{Z}}
\newcommand{\Q}{\mathbb{Q}}
\newcommand{\R}{\mathbb{R}}

\newtheorem{thm}{Teorema}[chapter]
\newtheorem{lem}[thm]{Lema}
\newtheorem{cor}[thm]{Corolário}
\newtheorem{thm*}[thm]{Teorema*}
\newtheorem{lem*}[thm]{Lema*}
\newtheorem{cor*}[thm]{Corolário*}

\DeclareMathOperator{\sen}{sen}
\DeclareMathOperator{\arcsen}{arc sen}
\DeclareMathOperator{\senh}{senh}
\DeclareMathOperator{\tr}{tr}

\title{Notas de álgebra linear}
\author{Max Jáuregui}

%\linespread{1.2}
%\linespread{1.213} %11pt
%\linespread{1.241} %12pt

\begin{document}
\maketitle
  Estas notas foram criadas principalmente para meu uso pessoal e pode eventualmente ser usado como um curso elementar de álgebra linear. Todo o conteúdo foi produzido por mim, seguindo como roteiro o livro: E. L. Lima, \textit{Álgebra linear}, 8 ed. (IMPA, Rio de Janeiro, 2012).

  \begin{flushright}
    Max Jáuregui
  \end{flushright}
\tableofcontents

\chapter{Matrizes}

Uma \tb{matriz} de \tb{ordem} $m\times n$ é um arranjo retangular de $mn$ números reais ou complexos que tem $m$ linhas e $n$ colunas. Se $\mb{a}$ é uma matriz $m\times n$, então
$$\mb{a}=\begin{bmatrix}
  a_{11}&\ldots&a_{1n}\\
  \vdots&\vdots&\vdots\\
  a_{m1}&\ldots&a_{mn}\\
\end{bmatrix}\,,$$
em que, para cada $i\in\{1,\ldots,m\}$ e $j\in\{1,\ldots,n\}$, o número $a_{ij}$ é chamado de \tb{elemento} na posição $(i,j)$ da matriz $\mb{a}$.

Formatos especiais de matrizes:

\begin{enumerate}
  \item \tb{Matriz quadrada:} É uma matriz cujo número de linhas é igual ao seu número de colunas. Se $\mb{a}$ é uma matriz quadrada, os elementos $a_{ii}$ são chamados de elementos da \tb{diagonal} de $\mb{a}$. Exemplos:
  $$\begin{bmatrix}
    4&0\\
    2/3&-1
  \end{bmatrix}\,,\quad \begin{bmatrix}
    5&-2&1\\
    0&0&10\\
    2&1&\pi
  \end{bmatrix}\,.$$
  Os elementos da diagonal dessas matrizes são $(4,-1)$ e $(5,0,\pi)$ respectivamente.
  \item \tb{Matriz diagonal:} É uma matriz quadrada cujos elementos fora da diagonal são nulos. Exemplos:
  $$\begin{bmatrix}
    4&0\\
    0&-1
  \end{bmatrix}\,,\quad \begin{bmatrix}
    5&0&0\\
    0&0&0\\
    0&0&\pi
  \end{bmatrix}\,.$$
  \item \tb{Matriz identidade:} É uma matriz diagonal na qual todos os elementos da diagonal são iguais a $1$. A matriz identidade $n\times n$ é denotada por $\mb{I}_n$. Exemplos:
  $$\mb{I}_2=\begin{bmatrix}
    1&0\\
    0&1
  \end{bmatrix}\,,\quad \mb{I}_3=\begin{bmatrix}
    1&0&0\\
    0&1&0\\
    0&0&1
  \end{bmatrix}\,.$$
  \item \tb{Matriz triangular superior:} É uma matriz quadrada na qual todos os elementos debaixo da diagonal são nulos. Exemplos:
  $$\begin{bmatrix}
    4&3\\
    0&-1
  \end{bmatrix}\,,\quad \begin{bmatrix}
    5&3&0\\
    0&0&2\\
    0&0&\pi
  \end{bmatrix}\,.$$
  \item \tb{Matriz triangular inferior:} É uma matriz quadrada na qual todos os elementos acima da diagonal são nulos. Exemplos:
  $$\begin{bmatrix}
    4&0\\
    3&-1
  \end{bmatrix}\,,\quad \begin{bmatrix}
    5&0&0\\
    3&0&0\\
    0&2&\pi
  \end{bmatrix}\,.$$
  \item \tb{Matriz simétrica:} É uma matriz quadrada $\mb{a}$ tal que $a_{ij}=a_{ji}$ para quaisquer $i$ e $j$, ou seja, é uma matriz simétrica em relação à sua diagonal. Exemplos:
  $$\begin{bmatrix}
    4&3\\
    3&-1
  \end{bmatrix}\,,\quad \begin{bmatrix}
    5&3&0\\
    3&0&2\\
    0&2&\pi
  \end{bmatrix}\,.$$
  \item \tb{Matriz antissimétrica:} É uma matriz quadrada $\mb{a}$ tal que $a_{ij}=-a_{ji}$ para quaisquer $i$ e $j$. Segue daqui que $a_{ii}=-a_{ii}$ para todo $i$, o que implica que os elementos da diagonal de $\mb a$ são nulos. Exemplos:
  $$\begin{bmatrix}
    0&-3\\
    3&0
  \end{bmatrix}\,,\quad \begin{bmatrix}
    0&3&0\\
    -3&0&2\\
    0&-2&0
  \end{bmatrix}\,.$$
  \item \tb{Matriz hermitiana:} É uma matriz quadrada $\mb a$ tal que $a_{ij}=\overline{a_{ji}}$ para quaisquer $i$ e $j$, em que $\overline{a_{ji}}$ denota o complexo conjugado de $a_{ji}$ ($\overline{x+iy}=x-iy$, em que $i=\sqrt{-1}$). Segue daqui que $a_{ii}=\overline{a_{ii}}$ para todo $i$, o que implica que os elementos da diagonal de $\mb a$ são reais. Exemplos:
  $$\begin{bmatrix}
    4&3i\\
    -3i&-1
  \end{bmatrix}\,,\quad \begin{bmatrix}
    5&3+i&0\\
    3-i&0&-2i\\
    0&2i&\pi
  \end{bmatrix}\,.$$
\end{enumerate}

Como veremos mais na frente, o conjunto de todas as matrizes $m\times n$ é um espaço vetorial, pois nele podem ser definidas duas operações: adição de matrizes e multiplicação de uma matriz por um número (real ou complexo dependendo das circunstâncias). As seguintes equações definem essas operações:
$$\mb{a}+\mb{b}=\begin{bmatrix}
  a_{11}+b_{11}&\ldots&a_{1n}+b_{1n}\\
  \vdots&\vdots&\vdots\\
  a_{m1}+b_{m1}&\ldots&a_{mn}+b_{mn}
\end{bmatrix}\,,\quad
\alpha\mb{a}=\begin{bmatrix}
  \alpha a_{11}&\ldots&\alpha a_{1n}\\
  \vdots&\vdots&\vdots\\
  \alpha a_{m1}&\ldots&\alpha a_{mn}
\end{bmatrix}\,.$$
Pode-se verificar facilmente que essas operações têm as seguintes propriedades:
\begin{enumerate}
  \item Associatividade: $\mb{a}+(\mb{b}+\mb{c})=(\mb{a}+\mb b)+\mb c$ e $\alpha(\beta \mb a)=(\alpha\beta)\mb a$.
  \item Comutatividade: $\mb a+\mb b=\mb b+\mb a$.
  \item Elemento neutro: Existe uma matriz $\mb 0$ tal que $\mb a+\mb 0=\mb a$ para qualquer $\mb a$.
  \item Elemento inverso: Para cada $\mb a$ existe uma matriz $-\mb a$ tal que $\mb a+(-\mb a)=\mb 0$.
  \item Multiplicação por $1$: $1\mb a=\mb a$.
  \item Distributividade: $\alpha(\mb a+\mb b)=\alpha\mb a+\alpha\mb b$ e $(\alpha+\beta)\mb a=\alpha\mb a+\beta\mb a$.
\end{enumerate}

Define-se a \tb{transposta} de uma matriz $\mb a$ como a matriz $\mb a^T$ cujas linhas são as colunas de $\mb a$, ou seja, $a_{ij}^T=a_{ji}$ para quaisquer $i$ e $j$. Segue diretamente dessa definição que $(\mb a^T)^T=\mb a$, $(\mb a+\mb b)^T=\mb a^T+\mb b^T$ e $(\alpha\mb a)^T=\alpha\mb a^T$.

Exemplos:

\begin{enumerate}
  \item Uma matriz $\mb a$ é simétrica (antissimétrica) se, e somente se, $\mb a=\mb a^T$ ($\mb a=-\mb a^T$).
  \item Uma matriz $\mb a$ é hermitiana se, e somente se, $\mb a^T=\overline{\mb a}$, em que $\overline{\mb a}$ é a matriz obtida ao tomar o complexo conjugado de cada elemento de $\mb a$.
  \item Toda matriz quadrada $\mb a$ pode ser escrita como a soma de uma matriz simétrica e uma matriz antissimétrica. Com efeito, definindo $\mb s=(\mb a+\mb a^T)/2$ e $\mb t=(\mb a-\mb a^T)/2$, temos que $\mb s=\mb s^T$, $\mb t=-\mb t^T$ e $\mb a=\mb s+\mb t$.
\end{enumerate}

Além das duas operações anteriores, pode ser definida uma \tb{multiplicação de matrizes}. Se $\mb a$ é uma matriz $m\times n$ e $\mb b$ é uma matriz $n\times p$, então a multiplicação de $\mb a$ com $\mb b$ pode ser efetuada e o produto é uma matriz $\mb c$ de ordem $m\times p$ tal que
$$c_{ik}=\sum_{j=1}^na_{ij}b_{jk}=a_{i1}b_{1k}+\cdots+a_{in}b_{nk}\,.$$
Em outras palavras, o elemento $c_{ik}$ da matriz produto $\mb c$ é obtido multiplicando cada elemento da linha $i$ da matriz $\mb a$ com o elemento respectivo na coluna $k$ da matriz $\mb b$ e somando esses produtos.

Exemplos:
\begin{enumerate}
  \item Dadas as matrizes
  $$\mb a=\begin{bmatrix}
    1&0&-3\\
    2&1&0
  \end{bmatrix}\quad\text{e}\quad\mb b=\begin{bmatrix}
    1&5&0\\
    4&-1&3\\
    0&2&0
  \end{bmatrix}\,,$$
  temos que
  $$\mb {ab}=\begin{bmatrix}
    1&-1&0\\
    6&9&3
  \end{bmatrix}\,.$$
  No entanto, o produto $\mb{ba}$ não está definido, pois o número de linhas de $\mb a$ é diferente do número de colunas de $\mb b$.
  \item Se $\mb a$ é uma matriz $m\times n$, o produto de $\mb a$ com $\mb I_n$ é o próprio $\mb a$ (isso justifica o nome de matriz identidade). Com efeito, os elementos da matriz $\mb I_n$ são dados pelo \tb{símbolo de Kronecker}
  $$\delta_{ij}=\begin{cases}
    1&\text{se $i=j$}\\
    0&\text{se $i\ne j$.}
  \end{cases}$$
  Logo, se $\mb c=\mb {aI}_n$, então $c_{ik}=\sum_{j=1}^na_{ij}\delta_{jk}=a_{ik}\delta_{kk}=a_{ik}$. De forma análoga pode-se provar que $\mb I_m\mb a=\mb a$.
  \item Duas matrizes não nulas podem ter como produto a matriz nula. Com efeito,
  $$\begin{bmatrix}
    1&0&1\\
    0&-1&2
  \end{bmatrix}\begin{bmatrix}
    -1&3\\
    2&-6\\
    1&-3
  \end{bmatrix}=\begin{bmatrix}
    0&0\\
    0&0
  \end{bmatrix}\,.$$
  Por outro lado
  $$\begin{bmatrix}
    -1&3\\
    2&-6\\
    1&-3
  \end{bmatrix}\begin{bmatrix}
    1&0&1\\
    0&-1&2
  \end{bmatrix}=\begin{bmatrix}
    -1&-3&5\\
    2&6&-10\\
    1&3&-5
  \end{bmatrix}\,,$$
  o que mostra que a multiplicação de matrizes não é comutativa.
  \item Mesmo a multiplicação de duas matrizes quadradas não é comutativa, pois, por exemplo,
  $$\begin{bmatrix}
    2&3\\
    -1&0
  \end{bmatrix} \begin{bmatrix}
    0&1\\
    1&0
  \end{bmatrix}=\begin{bmatrix}
    3&2\\
    0&-1
  \end{bmatrix}\quad\text{e}\quad\begin{bmatrix}
    0&1\\
    1&0
  \end{bmatrix} \begin{bmatrix}
    2&3\\
    -1&0
  \end{bmatrix}= \begin{bmatrix}
    -1&0\\
    2&3
  \end{bmatrix}\,.$$
\end{enumerate}

\begin{thm}
  Sejam $\mb a$ e $\mb b$ matrizes $m\times n$, $\mb c$ e $\mb d$ matrizes $n\times p$ e $\alpha$ um número. Tem-se que $\mb a(\mb b+\mb c)=\mb {ab}+\mb{ac}$, $(\mb a+\mb b)\mb c=\mb {ac}+\mb{bc}$ e $\mb a(\alpha\mb b)=\alpha (\mb {ab})=(\alpha \mb a)\mb b$.
\end{thm}
\begin{proof}
  Se $\mb p=\mb a(\mb b+\mb c)$, então
  $$p_{ik}=\sum_{j=1}^na_{ij}(b_{jk}+c_{jk})=\sum_{j=1}^na_{ij}b_{jk}+\sum_{j=1}^na_{ij}c_{jk}\,,$$
  o que implica que $\mb p=\mb {ab}+\mb{ac}$. A igualdade $(\mb a+\mb b)\mb c=\mb {ac}+\mb{bc}$ pode ser provada de forma análoga.
\end{proof}

Exemplo: Dada a matriz
$$\mb a=\begin{bmatrix}
  1&x\\
  0&1
\end{bmatrix}\,,$$
temos que
$$\mb a^n=\begin{bmatrix}
  1&nx\\
  0&1
\end{bmatrix}$$
para todo $n\in\N$. Com efeito, definindo
$$\mb t=\begin{bmatrix}
  0&x\\
  0&0
\end{bmatrix}$$
temos que $\mb a=\mb I_2+\mb t$ e que $\mb t^2=\mb 0$. Logo,
$$\mb a^2=(\mb I_2+\mb t)^2=(\mb I_2+\mb t)(\mb I_2+\mb t)=\mb I_2^2+\mb I_2\mb t+\mb t\mb I_2+\mb t^2=\mb I_2+2\mb t\,.$$
Supondo que $\mb a^n=\mb I_2+n\mb t$ para algum $n\in\N$, temos que
$$\mb a^{n+1}=\mb a^n\mb a=(\mb I_2+n\mb t)(\mb I_2+\mb t)=\mb I_2^2+\mb I_2\mb t+n\mb t\mb I_2+n\mb t^2=\mb I_2+(n+1)\mb t\,.$$
Portanto, $\mb a^n=\mb I_2+n\mb t$ para todo $n\in\N$.

\begin{thm}
  Sejam $\mb a$ uma matriz $m\times n$, $\mb b$ uma matriz $n\times p$ e $\mb c$ uma matriz $p\times q$. Tem-se que $\mb a(\mb {bc})=(\mb{ab})\mb c$.
\end{thm}
\begin{proof}
  Sejam $\mb r=\mb{ab}$, $\mb s=\mb {bc}$ e $\mb t=\mb {as}$. Temos que
  $$t_{ik}=\sum_{j=1}^na_{ij}s_{jk}=\sum_{j=1}^na_{ij}\sum_{l=1}^pb_{jl}c_{lk}=\sum_{j=1}^n\sum_{l=1}^pa_{ij}b_{jl}c_{lk}=\sum_{l=1}^p\dpar{\sum_{j=1}^na_{ij}b_{jl}}c_{lk}=\sum_{l=1}^pr_{il}c_{lk}\,.$$
  Portanto, $\mb t=\mb {rc}$.
\end{proof}

\begin{thm}
  Sejam $\mb a$ uma matriz $m\times n$ e $\mb b$ uma matriz $n\times p$. Tem-se que $(\mb a\mb b)^T=\mb b^T\mb a^T$.
\end{thm}
\begin{proof}
  Se $\mb c=\mb {ab}$, então
  $$c^T_{ik}=c_{ki}=\sum_{j=1}^na_{kj}b_{ji}=\sum_{j=1}^na_{jk}^Tb_{ij}^T=\sum_{j=1}^nb_{ij}^Ta_{jk}^T\,.$$
  Portanto, $\mb c^T=\mb b^T\mb a^T$.
\end{proof}

Define-se o \tb{traço} de uma matriz quadrada $\mb a$ como a soma dos elementos da sua diagonal. O traço de $\mb a$ é denotado por $\tr(\mb a)$. Segue imediatamente dessa definição que $\tr(\mb a)=\tr(\mb a^T)$, pois $\mb a$ e $\mb a^T$ têm a mesma diagonal. Além disso, $\tr(\mb a+\mb b)=\tr(\mb a)+\tr(\mb b)$ e $\tr(\alpha\mb a)=\alpha\tr(\mb a)$.

\begin{thm}
  Sejam $\mb a$ uma matriz $m\times n$ e $\mb b$ uma matriz $n\times m$. Tem-se que $\tr(\mb {ab})=\tr(\mb{ba})$.
\end{thm}
\begin{proof}
  Se $\mb c=\mb a\mb b$, então $\tr(\mb {ab})=\sum_{i=1}^m c_{ii}$. Como $c_{ii}=\sum_{j=1}^na_{ij}b_{ji}$, segue que
  $$\tr(\mb{ab})=\sum_{i=1}^m\sum_{j=1}^na_{ij}b_{ji}=\sum_{j=1}^n\sum_{i=1}^ma_{ij}b_{ji}=\sum_{j=1}^n\sum_{i=1}^mb_{ji}a_{ij}\,.$$
  Portanto, $\tr(\mb{ab})=\tr(\mb{ba})$.
\end{proof}

Diz-se que uma matriz é \tb{escalonada} quando o primeiro elemento não-nulo de cada linha está à esquerda dos primeiros elementos não-nulos das linhas subsequentes. Dessa maneira, debaixo do primeiro elemento não-nulo de cada linha só se tem zeros. Exemplos:
$$\begin{bmatrix}
  2&-1&0&4\\
  0&2&3&-1\\
  0&0&0&5
\end{bmatrix}\,,\quad \begin{bmatrix}
  0&1&2&3&4\\
  0&0&0&3&5\\
  0&0&0&0&1\\
  0&0&0&0&0
\end{bmatrix}\,.$$

Vamos agora descrever o método de eliminação de Gauss que nos permite transformar uma matriz qualquer em uma matriz escalonada. O método de eliminação consiste em usar sistematicamente as seguintes operações nas linhas da matriz até obter uma matriz escalonada:
\begin{enumerate}
  \item permutar duas linhas;
  \item multiplicar uma linha por um número diferente de $0$;
  \item somar a uma linha um múltiplo não-nulo de uma outra linha.
\end{enumerate}
Essas operações são às vezes chamadas de \tb{operações elementares}.

Exemplo:
\begin{multline*}
  \begin{bmatrix}
    0&0&2&0\\
    0&1&3&1\\
    0&3&0&2
  \end{bmatrix}\xrightarrow{L_1\leftrightarrow L_2} \begin{bmatrix}
    0&1&3&1\\
    0&0&2&0\\
    0&3&0&2
\end{bmatrix}\xrightarrow{L_3-3L_1} \begin{bmatrix}
  0&1&3&1\\
  0&0&2&0\\
  0&0&-9&-1
\end{bmatrix}\xrightarrow{L_2/2}\begin{bmatrix}
  0&1&3&1\\
  0&0&1&0\\
  0&0&-9&-1
\end{bmatrix}\\
\xrightarrow{L_3+9L_2}\begin{bmatrix}
  0&1&3&1\\
  0&0&1&0\\
  0&0&0&-1
\end{bmatrix}\,.
\end{multline*}

Define-se o \tb{posto} de uma matriz como o número de linhas não-nulas da matriz escalonada obtida usando o método de eliminação. Segue imediatamente daqui que o posto de uma matriz $m\times n$ é no máximo igual a $m$. O posto de uma matriz não depende dos detalhes do processo de eliminação (isso será justificado mais na frente). Por exemplo, a matriz considerada no exemplo anterior tem posto $3$.

Diz-se que uma matriz $\mb a$ de ordem $n\times n$ é \tb{invertível} se existe uma matriz $\mb a^{-1}$ de ordem $n\times n$, chamada de \tb{inversa} da matriz $\mb a$, tal que $\mb a\mb a^{-1}=\mb a^{-1}\mb a=\mb I_n$. Como será justificado mais na frente, uma matriz quadrada é invertível se, e somente se, seu posto é igual ao seu número de colunas (ou linhas).

Exemplos:
\begin{enumerate}
  \item A matriz
  $$\mb a=\begin{bmatrix}
    1&2&3\\
    4&5&6\\
    7&8&9
  \end{bmatrix}$$
  não é invertível. Com efeito, temos que
  $$\begin{bmatrix}
    1&2&3\\
    4&5&6\\
    7&8&9
  \end{bmatrix}\xrightarrow[L_3-7L_1]{L_2-4L_1} \begin{bmatrix}
    1&2&3\\
    0&-3&-6\\
    0&-6&-12
  \end{bmatrix}\xrightarrow{L3-2L_2} \begin{bmatrix}
    1&2&3\\
    0&-3&-6\\
    0&0&0
  \end{bmatrix}\,,$$
  do qual segue que o posto de $\mb a$ é $2\ne 3$.
\end{enumerate}

\end{document}
