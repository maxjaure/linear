\documentclass[12pt,a4paper]{article}
\usepackage[portuguese]{babel}
\usepackage[margin=1in]{geometry}
\usepackage[utf8]{inputenc}
\usepackage{amsmath}
\usepackage{amsthm}
\usepackage{amssymb}
\usepackage{amsfonts}
\usepackage[colorlinks,allcolors=blue]{hyperref}
\usepackage{microtype}
\usepackage{graphicx}

\let\emptyset=\varnothing

\newcommand{\tb}{\textbf}
\newcommand{\tbu}[1]{\tb{\textup{#1}}}
\newcommand{\mb}{\mathbf}
\newcommand{\mc}{\mathcal}

\newcommand{\dpar}[1]{\left(#1\right)}
\newcommand{\dsqr}[1]{\left[#1\right]}
\newcommand{\dcur}[1]{\left\{#1\right\}}
\newcommand{\dabs}[1]{\left|#1\right|}
\newcommand{\ang}[1]{\left\langle#1\right\rangle}

\newcommand{\ds}{\displaystyle}

\newcommand{\N}{\mathbb{N}}
\newcommand{\Z}{\mathbb{Z}}
\newcommand{\Q}{\mathbb{Q}}
\newcommand{\R}{\mathbb{R}}

\DeclareMathOperator{\sen}{sen}
\DeclareMathOperator{\arcsen}{arc sen}
\DeclareMathOperator{\senh}{senh}
\DeclareMathOperator{\tr}{tr}
\DeclareMathOperator{\posto}{posto}
\DeclareMathOperator{\sgn}{sgn}

\title{Primeiro teste de álgebra linear}
\date{}
%\linespread{1.2}
%\linespread{1.213} %11pt
%\linespread{1.241} %12pt

\begin{document}
\maketitle
\begin{enumerate}
  \item ($0{,}25$pts) Escreva uma matriz hermitiana $3\times 3$ cujos elementos sejam todos não-nulos mas que não sejam todos reais.
  \item ($0{,}25$pts) Escreva uma matriz triangular superior $4\times 4$ não diagonal cujo traço seja $5$.
  \item ($0,25$pts) Escreva a matriz $\mb a$ de ordem $3\times 3$ cujos elementos são dados por
  $$a_{ij}=(i^2+j^2-1)\delta_{ij}\,,$$
  em que $\delta_{ij}$ é o símbolo de Kronecker.
  \item ($0{,}75$pts) Dadas as matrizes
  $$\mb a=\begin{bmatrix}
    4&5&3\\
    -2&1&0
  \end{bmatrix}\,,\quad \mb b=\begin{bmatrix}
    0&-2&0\\
    2&0&2
  \end{bmatrix}\quad\text{e}\quad \mb c=\begin{bmatrix}
      0&-2&0\\
      -1&0&1
  \end{bmatrix}\,,$$
  calcule (a) $2\mb a-\mb b+3\mb c$, (b) $\mb b\mb a^T$, (c) $\mb a^T\mb b+\mb a^T\mb c$.%, (d) $\mb a^T\mb b+\mb b^T\mb a$, (e) $\mb{ab}^T\mb c$.
  \item ($0{,}75$pts) Usando o método de eliminação mostre que o seguinte sistema linear tem a lista $(1,1,1)$ como a sua única solução:
  \begin{equation*}
    \begin{split}
      3x-y+z&=3\\
      -x+5y-z&=3\\
      x+2y-z&=2\,.
    \end{split}
  \end{equation*}
  \item ($0{,}75$pts) Determine a inversa da matriz
  $$\begin{bmatrix}
      6&2&5\\
      2&1&2\\
      -1&2&2
  \end{bmatrix}\,.$$
\end{enumerate}
\end{document}
